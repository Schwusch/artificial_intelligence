\documentclass{article}
\usepackage[utf8]{inputenc}

\title{AI Othello Assignment}
\author{Jonathan Böcker}
\date{September 2016}

\usepackage{natbib}
\usepackage{graphicx}
\usepackage{listings}
\usepackage{color}
\usepackage[a4paper, margin=1in]{geometry}

\definecolor{codegreen}{rgb}{0,0.6,0}
\definecolor{codegray}{rgb}{0.5,0.5,0.5}
\definecolor{codepurple}{rgb}{0.58,0,0.82}
\definecolor{backcolour}{rgb}{0.95,0.95,0.92}

\lstdefinestyle{mystyle}{
    backgroundcolor=\color{backcolour},
    commentstyle=\color{codegreen},
    keywordstyle=\color{magenta},
    numberstyle=\tiny\color{codegray},
    stringstyle=\color{codepurple},
    basicstyle=\footnotesize,
    breakatwhitespace=false,
    breaklines=true,
    captionpos=b,
    keepspaces=true,
    numbers=left,
    numbersep=5pt,
    showspaces=false,
    showstringspaces=false,
    showtabs=false,
    tabsize=2
}


\lstset{style=mystyle}

\begin{document}

\maketitle

\section{Introduction}
This assignment consists of implementing the alpha-beta pruning search algorithm
and heuristic evaluation functions for an Othello AI opponent. I chose to implement
proper Othello rules and expand the board area to an 8 by 8 grid to make it a little
more challenging. I also chose to make the search algorithm with a iterative deepening search,
making it time dependent instead.
This makes the algorithm `smarter' the faster the host CPU is.

\section{Building the search algorithm}
I chose to represent the search nodes as states, that holds a snapshot of all important
variables in the state. It is important to understand this class in order to understand
the search algorithm:

\begin{lstlisting}[language=Java]
class StateNode {
    private int[][] gridState;
    private int alpha = Integer.MIN_VALUE;
    private int beta = Integer.MAX_VALUE;
    private int player;
    private boolean isEndState = false;
    private int depth;
    private long deadline;
    private StateChange bestChange;
}
\end{lstlisting}

\begin{itemize}
\item The \verb|gridState| variable is an integer matrix where values span from -1 to 1,
where -1 represents a human player brick, 1 represents an AI brick and 0 represents
an unoccupied space. This is a application wide representation the Othello board.

 \item The \verb|alpha| and \verb|beta| variables is important to the pruning algorithm at
every node to evaluate the need of searching further down in the current branch.

\item The \verb|player| int variable is either 1(AI) or -1(Human) and tells the algorithm
which player is about to make the next virtual draw in this node.

\item The \verb|isEndState| boolean variable is used to determine if a total board score
is relevant. This is important when the algorithm needs to know if the node is a
leafnode e.g. when the algorithm isn't allowed to search further due to time constraints.

\item The \verb|depth| int variable is only for assignment requirements sake. It keeps
track of current search depth, making it possible to graphically display the search depth for the user.

\item The \verb|deadline| long integer variable is a timestamp, for when the search
algorithm must stop searching in this branch in order to uphold the 5 second
draw restraint.

\item The last instance variable \verb|bestChange| is an object with the following specifications:
\end{itemize}

\begin{lstlisting}[language=Java]
class StateChange {
    private StateNode endNode;
    private OthelloCoordinate move;
}
\end{lstlisting}

Its purpose is to keep track of the, according to the search algorithm, best state
change and by extension the best next draw for this node. This is most useful
for the root node when asked to provide the next AI draw. In this context the draw
is an object named move, simply holding a row and a column value in the board matrix as follows:
\begin{lstlisting}[language=Java]
class OthelloCoordinate {
    private int row;
    private int col;
}
\end{lstlisting}

\verb|StateNode| has one package-private constructor which is called when the application
controller wants a calculated draw from the AI:

\begin{lstlisting}[language=Java]
StateNode(int[][] gridState, OthelloController controller, long deadline){
      this.gridState = gridState;
      this.player = OthelloGUI.AI;
      this.depth = 0;
      this.deadline = deadline;
      controller.nodeFound(this.depth);
      findAllStateChanges(controller);
}
\end{lstlisting}

The constructor passes on the \verb|OthelloController| to the recursive method
\verb|findAllStateChanges()| after it has reported itself as a found node at depth 0.
One of the first things done in \verb|findAllStateChanges()| is checking if it should
attempt to find more nodes:

\begin{lstlisting}[language=Java]
private void findAllStateChanges(OthelloController controller){
      ArrayList<StateChange> possibleChanges = new ArrayList<>();
      // Check if there is any limitations or we should keep looking
      if(this.shouldKeepLooking()) {
      ...
\end{lstlisting}

The method \verb|shouldKeepLooking()| has the purpose of limiting the search depth
according to some criteria, in this case it's a time limit:

\begin{lstlisting}[language=Java]
private boolean shouldKeepLooking() {
        return System.currentTimeMillis() < this.deadline;
}
\end{lstlisting}

If \verb|shouldKeepLooking()| returns true, \verb|findAllStateChanges()| will continue to take
all possible draws and create a new \verb|StateNode| for each one using the private
constructor solely for recursive calls:

\begin{lstlisting}[language=Java]
private StateNode(int[][] gridState, int player, int alpha, int beta, int depth, OthelloController controller, long deadline)
\end{lstlisting}

The private constructor requires a little more information. The \verb|player|
variable oscillates between Human or AI between every recursive call in order for the
pruning algorithm to decide whether it's a prunable branch or not. The player variable
also controls if alpha or beta is passed back to the parent node when asked for its value.

\section{The pruning condition}
The pruning algorithms reasoning is that if the current node is AI, the child node
must be human. If that's so, it wants to see if the child nodes beta value is larger
than its own alpha value. Because if it is, that path of draws is the most favorable
for the moment, so it saves that draw and beta value as its alpha value until it finds a better one.

In the other way around, when the current node is human and child node is AI, it
checks if its child nodes alpha value is smaller than its beta.
If that's true it updates its beta to the child nodes alpha and saves the draw
as the most favorable draw. If the child node is a leaf node, it simply compares with the board score count.

Lastly, the condition that decides whether there's reason to keep looking in that branch,
is if the current node is AI and its alpha is larger than its beta. If that's true there is
no point in looking further into that branch.
In the other way around, if the node is human and its alpha is smaller than its beta, there is
no reason to continue in that branch.

\section{The evaluation functions}
\end{document}
